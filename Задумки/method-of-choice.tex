Глава с краткими примерами раздумий при выборе метода диагностики.

Например:
Девочка 17 лет с подозрением на синдром Титце - МРТ - долго и дискофортно лежать на животе, зачем делать КТ (облучение молодого организма), если можно сделать УЗИ с Power Doppler и получить усиление кровтока вокруг хрящей, как признак активности воспалительного процесса.

Пациентка с травмой инфраорбитальной области в анамнезе с наличием парестезий данной области. При разговоре со врачом-стоматологом - не проверена типичность выхода инфраорбитальной ветви 5го нерва, не выполнено УЗИ (высокочастотным датчиком, с применением допплера) для оценки расположения и плотности структур в интересующей области.