\section{Магнитнорезонанская томография}

\subsection{Физические принципы}

\subsection{Показания и противопоказания}

\begin{enumerate}
\item МРТ и кардиостимуляторы
\item МРТ и протезы, импланты, клапаны
\end{enumerate}


\subsection{А все-таки ЯМР или МРТ?}
Как вы наверное знаете, оба этих термина --- ЯМР и МРТ встречаются как в литературе, так и в разговорах и назначениях врачей-специалистов. Разберемся, в чем же разница и как правильно.

Начнем с расшифровки и определения терминов. ЯМР -- ядерный магнитный резонанс -- поглощение и излучение элетромагнитной энергии веществом, содержащим ядра с ненулевым спином во внешнем магнитном поле. Итак, это явление, лежащее в основе рассматриваемого нами метода. Вы можете возразить, ведь существует так же и термин ЯМРТ -- то есть ЯМР-томография. Да, вы правы, но ведь во-первых он длинный (на целую букву длиннее), а во-вторых, давайте вспомним 1986 год.

1 мая 1986 года, страна Советов. Мир, труд, май! Проходят манифестации и парады. Страна и Мир не знают, что в ночь с 25 на 26 апреля произошла одна из самых страшных катастроф 20 века - авария на Чернобыльской АЭС. Весь Мир начинает бояться слов атом и ядерный. 

А на горизонте - новая, еще достаточно мало известная, многообещающая технология диагностики заболеваний. Связь с аварией только в слове <<ядерный>>, но вредностью -- не обладает. Так почему бы не убрать это самое ненужное в названии слово? Было ЯМРТ, стало -- МРТ. Про магнит -- мы сказали, резонанс -- в названии оставили, да еще и томографию -- то есть срезы, не забыли упомянуть. 

11 марта 2011 года - землетрясение, цунами, авария на АЭС Фукусима-1. Атомная, ядерная -- слова постоянно заменяющие друг-друга в репортажах корреспондентов. Опять мелькает название Чернобыля, опять развивается нуклеофобия у простого населения и недоверие к слову <<ядерный>>, если оно стоит рядом с безопасностью. В результате -- новая волна вопросов пациентов -- <<А разве МРТ не безопасней ЯМР?>> Как мы с вами видим -- вопрос, возникший из-за незнания, на котором растут предрассудки. 

Итого: оба названия правильные и имеют право на жизнь. Однако, в настоящий момент, считается более правильным название МРТ, и как в достаточной степени описывающее методику, и как менее раздражающее для общественности. Хотя фактор раздражения, в большей степени, уже утратил свою силу по простой причине -- способности человека забывать об ошибках и бедах прошлого. 

\subsection{Война Тесла}





