\section{Магнитнорезонанская томография}

\subsection{Физические принципы}
\epigraph{Каждая достаточно развитая технология не отличима от магии}{Артур Кларк}
\epigraph{Данная лекция обычно понимается студентами МГТУ и физического факультета МГУ с третьего раза. Медика, понявшего эту лекцию, в природе не встречено}{Панов}

\subsection{Показания и противопоказания}

\begin{enumerate}
\item МРТ и кардиостимуляторы
\item МРТ и протезы, импланты, клапаны
\end{enumerate}


\subsection{А все-таки ЯМР или МРТ?}
Как вы наверное знаете, оба этих термина --- ЯМР и МРТ встречаются как в литературе, так и в разговорах и назначениях врачей-специалистов. Разберемся, в чем же разница и как правильно.

Начнем с расшифровки и определения терминов. ЯМР --- ядерный магнитный резонанс --- поглощение и излучение элетромагнитной энергии веществом, содержащим ядра с ненулевым спином во внешнем магнитном поле. Итак, это явление, лежащее в основе рассматриваемого нами метода. Вы можете возразить, ведь существует так же и термин ЯМРТ --- то есть ЯМР-томография. Да, вы правы, но ведь во-первых он длинный (на целую букву длиннее), а во-вторых, давайте вспомним 1986 год.

1 мая 1986 года, страна Советов. Мир, труд, май! Проходят манифестации и парады. Страна и Мир не знают, что в ночь с 25 на 26 апреля произошла одна из самых страшных катастроф 20 века - авария на Чернобыльской АЭС. Весь Мир начинает бояться слов атом и ядерный. 

А на горизонте - новая, еще достаточно мало известная, многообещающая технология диагностики заболеваний. Связь с аварией только в слове <<ядерный>>, но вредностью --- не обладает. Так почему бы не убрать это самое ненужное в названии слово? Было ЯМРТ, стало --- МРТ. Про магнит --- мы сказали, резонанс --- в названии оставили, да еще и томографию, то есть срезы, не забыли упомянуть. 

11 марта 2011 года --- землетрясение, цунами, авария на АЭС Фукусима-1. Атомная, ядерная --- слова постоянно заменяющие друг-друга в репортажах корреспондентов. Опять мелькает название Чернобыля, опять развивается нуклеофобия у простого населения и недоверие к слову <<ядерный>>, если оно стоит рядом с безопасностью. В результате --- новая волна вопросов пациентов --- <<А разве МРТ не безопасней ЯМР?>> Как мы с вами видим -- вопрос, возникший из-за незнания, на котором растут предрассудки. 

Итого: оба названия правильные и имеют право на жизнь. Однако, в настоящий момент, считается более правильным название МРТ, и как в достаточной степени описывающее методику, и как менее раздражающее для общественности. Хотя фактор раздражения, в большей степени, уже утратил свою силу по простой причине --- способности человека забывать об ошибках и бедах прошлого. 

\subsection{Война Тесла}
\epigraph{---Почему у вас такие плохие изображения, у вас же аппарат 3Т?\\--- Да нет, всего лишь 0,4Т!\\---А почему же тогда у вас такие хорошие изображения?}

Никола Тесла... чем же он так провинился, что его именем названи целую войну. Хорошо известный Австро-Венгерский ученый, подаривший этому миру множество изобретений, победитель <<Войны Токов>>. Конечно, виноват не он, а единица магнитной индукции, названная его именем. 

Как вы уже наверное знаете, медицинские аппараты магнитно-резонансной томографии делятся по различным параметрам на классы, и в том числе, по такому, как сила напряженности магнитного поля. Таким образом мы имеем сверхнизкопольные аппараты (менее 0,1Т), низкопольные (до 0,1--0,5Т), среднепольные (0,5--1Т), высокопольные (1--3Т и более). Иногда выделяют класс сверхвысокопольных аппаратов (7Т, 9Т и более). Каждый из них имеет свои достоинства и недостатки, но почему-то нередко слышны возгласы --- <<исследование только на магните от 1,5Т>>. Тому есть несколько причин, начиная от обоснованных в виде возможностей аппарата, так и других, не достаточно обоснованных и связанных с особенностями Российского рынка медицниской диагностики и подготовки специалистов-лучевиков. Постараемся разобраться в причинах и посмотрим насколько обоснованы данные заявлениями. 

Маленькое отступление от наших рассуждений. Я, автор данных строк, не являюсь адептом ни одной из сторон данной войны, и у меня нет никакой коммерческой заинтересованности в выигрыше одной из них. Я за здравый смысл и взвешенный подход к назначению исследований и использованию доступных средств. Все изображения, представленные в этой главе, могут быть получены в виде анонимизированных DICOM снимков при обращении к автору.

В достаточно известной книге Питера Ринка\cite{Rinck.MagnResInMed.1993} данной войне посвящена отдельная интерлюдия. Там же дана отсылка, что война началась в начале 1980 годов, когда практически все аппараты были низкопольные и качество изображения на них была крайне далеко от привычного нам. Когда встал вопрос о его повышении, инженеры сказали однозначно --- повышайте напряженность магнитного поля. На разработанных аппаратах действительно резко выросло отношение сигнал/шум, но и разработчики низко- и среднепольных аппаратов не сидели на месте. Борясь за качество изображения они достигли многого.

Одной из причин роста напряженности поля была задумка использовать МР-спектроскопию in vivo, но данная идея не выстрелила и во многом была забыта, оставив за МРТ только визуализацию.

Дальше еще интереснее --- в 1983 году на конференции в Сан-Франциско, обсуждение проблемы начавшееся в зале, вышло за его пределы и практически переросло в драку на кулаках. Ответ об оптимуме так и не был найден. 

Напряженность магнитного поля росла. Появлялись новые аппараты, но было замечена одна очень интересная вещь: резкое повышение качества, получаемое в пробирке на исследовательских аппаратах, не возникало при исследованиях in vivo. Дело в том, что организм человека обладает множеством параметров, которых нет в пробирке: человек дышит, у него бьется сердце, он глотает, органы пульсируют. При этом возникли проблемы - необходимость гасить паразитные сигналы и артефакты, возникшие в результате. Те, кто работают на сверхвысокопольных аппаратах расскажут вам, чего стоит добиться <<белого>> и однородного ликвора вокруг спинного мозга в шейном отделе позвонончика. Более того --- было показано, что в отдельных клинических ситуациях, хотя и достаточно узких, повышение напряженности магнитного поля приводит к понижению качества изображения и стиранию контраста.

В Российских реалиях дурную шутку сыграл еще один известный факт, связанный с МРТ в целом и выскокопольными и низкопольными аппаратами в частности. Так называемое --- ускоренное получение изображений. Да, на любом аппарате МРТ можно выбрать между качеством изображения, и временем за которое оно получается. То есть --- мы можем выиграть время, уменьшив качество изображения. Но 1990ые годы наложили свой отпечаток --- диагноз <<Очень хочу денег>> накрепко поселился в умы очень многих людей. Директора платных клиник гонят врачей МРТ, чтобы те делали больше исследований в день, но это невозможно без снижения их качества. Только если на высокопольных аппаратах понижение качества приводило к изображениям еще читаемым, то на низкопольных --- к таким, которые сложно использовать в диагностическом процессе. А куда деваться обычным врачам-нейрохирургам, неврологам и другим специалистам? Им приходится направлять на <<аппарат от 1,5Т>>, чтобы иметь хотя бы какую-то гарантию получения читаемых и диагностически ценных снимков. 

Не буду спорить, что пространственное разрешение выше у высокопольных аппаратов, но так ли оно везде необходимо? В большинстве случаев, если врача не <<гонят за деньгами>>, то на низкопольных аппаратах можно получить достойные изображения и ставить диагнозы, <<которые на них поставить не реально>>. Для примера --- несколько изображений из моей практики, с указанием аппаратов, на которых они получены.

%TODO: Менингеома 3*6мм, корона и поперки с Алтуфьево, и еще пару сюда вспомнить и добавить.

Надеюсь, я смог вас убедить, что не в Тесла счастье, и даже не в их количестве, а в правильной расстановке приоритетов у начальства и врача, мозгах и глазах проводящего и описывающего исследование. Единственное но --- на момент написания данных строк --- не дай Бог вам подумать купить китайский аппарат! Китай хорошо копирует внешне, но внутреннее качество остается ниже плинтуса. Это проявляется как в самом аппарате и его надежности, так и в получаемых изображениях.

Для повседневной работы и исключения большинства рутинных вещей в подавляющем количестве случаев будет достаточен и низкопольный аппарат. Однако, исследования, связанные с МР-спектроскопией, функциональным МРТ, МР-микроскопией, тензорграфией, а так же при необходимости различить отдельные мелкие детали --- потребуется мощный аппарат. Большинству специалистов, я бы порекомендовал просматривать снимки и описания пациентов на предмет центров и врачей от которых они приходят. Найдите для себя и выпишите 3--4 таких центра, которые вас устраивают по качеству изображений и описаний и рекомендуйте их. Не стоит пациентам говорить про магнитное поле, им это нужно меньше чем нам с вами. Им нужна - наша помощь как диагностов, так и лечащих врачей.

В заключение - диалог из эпиграфа действительно имел место быть, и следующей фразой было - <<У нас в клинике на 1,5Т не всегда такие получаются>>. 