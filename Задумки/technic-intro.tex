В данной главе, мы не хотим вам рассказывать обо всех тонкостях и подводных камнях всех известных вам методик лучевой диагностики. Достаточно сказать, что даже такой, кажись невзрачный, хотя и непонятный, метод диагностики, как МРТ имеет более 6 (да, шести) Нобелевских премий только за физику процесса. Что же мы скажем про простых врачей-диагностов, которые работают с этим и другими методами? Вы думаете они все это знают? К большому сожалению, даже они, зачастую, только догадываются о том,  что происходит в их аппаратах. Конечно, и достаточно часто, это приводит к неправильной диагностике, или неумению достоверно оценить результаты исследований.

Но мы не будем говорить о грустном и, в кратком и доступном для всех изложении, перескажем основные принципы работы всех лучевых методов исследований.