В данной главе, мы не хотим вам рассказывать обо всех тонкостях и подводных камнях всех известных вам методик лучевой диагностики. Достаточно сказать, что даже такой, кажись невзрачный, хотя и непонятный, метод диагностики, как МРТ имеет более 4 (да, четырех) Нобелевских премий только за физику процесса. Что же мы скажем про простых врачей-диагностов, которые работают с этим и другими методами? Вы думаете они все это знают? К большому сожалению, даже они, зачастую, только догадываются о том,  что происходит в их аппаратах. Конечно, и достаточно часто, это приводит к неправильной диагностике, или неумению достоверно оценить результаты исследований.

Но мы не будем говорить о грустном и, в кратком и доступном для всех изложении, перескажем основные принципы работы всех лучевых методов исследований.

\section{А лучевая ли диагностика?}

Мы хотели бы начать рассказ о физических принципах работы методов исследования с обсуждения термина <<Лучевая диагностика>>. Является ли она лучевой или нет? А может надо как-то по-другому ее называть?

<<Луч>> --- линия, вдоль которой переносится (световая, электромагнитная) энергия. Если совсем просто - пучок света малого поперечного размера. Но далее мы с вами убедимся, что далеко не все методы, с которыми мы сталкиваемся, используют именно луч. Иногда, термин луч вообще притянут за уши и, с технической точки зрения, не может быть использован. 

%TODO: перефразировал абзац ниже, надо проверить на подопытных.
Строго говоря - истинно лучевыми остаются только рентген и рентгеновская компьютерная томография. Условно лучевыми можно назвать методы, регистрирующие излучение от тела (Сцинтиграфия, ОФЭКТ, ПЭТ, Термография). А остальные? Локационные (зондирующие) --- УЗИ, с учетом оптической дифракции --- ОКТ (оптическая когерентная томография). Основывающиеся на методах переизлучения энергии, поглощенной телом --- МРТ. 

Получается, что <<лучевой>> диагностика остается чисто традиционно, не отражая современного положения дел. Спор о правомерности применения этого термина зашел в свое время в русскоязычном разделе Википедии. Спорящими сторонами оказались с одной стороны медицинские работники и к ним приближенные, с другой --- физики, связанные с медицинской аппаратурой. Итогом стал компромиссный термин --- <<Медицинская визуализация>>.
%Можно вставить вот эту ссылочку, хотя не знаю. (\\href{http://ru.wikipedia.org/wiki/\%D0\%9E\%D0\%B1\%D1\%81\%D1\%83\%D0\%B6\%D0\%B4\%D0\%B5\%D0\%BD\%D0\%B8\%D0\%B5_\%D0\%BA\%D0\%B0\%D1\%82\%D0\%B5\%D0\%B3\%D0\%BE\%D1\%80\%D0\%B8\%D0\%B8:\%D0\%9C\%D0\%B5\%D0\%B4\%D0\%B8\%D1\%86\%D0\%B8\%D0\%BD\%D1\%81\%D0\%BA\%D0\%B0\%D1\%8F_\%D0\%B2\%D0\%B8\%D0\%B7\%D1\%83\%D0\%B0\%D0\%BB\%D0\%B8\%D0\%B7\%D0\%B0\%D1\%86\%D0\%B8\%D1\%8F}{http://ru.wikipedia.org/wiki/Обсуждение_категории:Медицинская_визуализация}).

Стоит так же отметить, что в зарубежной литературе и научной среде используется термин Радиология, включающая все перечисленные выше методы исследований.
